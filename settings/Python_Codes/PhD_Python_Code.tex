\documentclass[a4paper,8pt]{article}
\usepackage[utf8]{inputenc}

%Page Geometry
\usepackage[
top=3cm,
bottom=2cm,
left=0.5cm,
right=0.5cm,
%bindingoffset=2cm, %Adds a Binding offset for printed documents
marginparwidth=6cm, %Adds Overflow Width to Rigth Margin 
marginparsep=3mm %Gap between right margin and right overflow margin
]{geometry}
\sloppy

% Default fixed font does not support bold face
\DeclareFixedFont{\ttb}{T1}{txtt}{bx}{n}{12} % for bold
\DeclareFixedFont{\ttm}{T1}{txtt}{m}{n}{12}  % for normal

% Custom colors
\usepackage{color}
\definecolor{deepblue}{rgb}{0,0,0.5}
\definecolor{deepred}{rgb}{0.6,0,0}
\definecolor{deepgreen}{rgb}{0,0.5,0}

\usepackage{listings}

% Python style for highlighting
\newcommand\pythonstyle{\lstset{
language=Python,
basicstyle=\ttm,
otherkeywords={self},             % Add keywords here
keywordstyle=\ttb\color{deepblue},
emph={MyClass,__init__},          % Custom highlighting
emphstyle=\ttb\color{deepred},    % Custom highlighting style
stringstyle=\color{deepgreen},
frame=tb,                         % Any extra options here
showstringspaces=false            % 
}}


% Python environment
\lstnewenvironment{python}[1][]
{
\pythonstyle
\lstset{#1}
}
{}

% Python for external files
\newcommand\pythonexternal[2][]{{
\pythonstyle
\lstinputlisting[#1]{#2}}}

% Python for inline
\newcommand\pythoninline[1]{{\pythonstyle\lstinline!#1!}}

\begin{document}

\section{Python Code}
\subsection{Calculation Codes}
\subsubsection{At\% to Wt\% Calculations}


\begin{python}
# In[1]:

'''Code for converting from At% to Wt% of an alloy
Code by Scott Gleason, University of New South Wales (UNSW), Australia 
S.Gleason@student.unsw.edu.au, April 2015'''

'''Takes an At% alloy and converts to Wt% for constituent elements
Element properties are entered as an object array
Prints alloy in At%, Wt%, expected density, 
and weight of each con element and total alloy wt.

From weights of con elements calcs Wt%, and returns At% & expected density'''


# In[2]:

'''Convert at% to wt% 
C1 = C1' * A1 / (C1' * A1 + ... + Cn' * An)

Convert wt% to at%
C1' = C1 / A1 / (C1 / A1 + ... + Cn / An)

Density from wt%
1 / D = C1 / D1 + ... + Cn / Dn

Where Cn = wt%, Cn' = at%, An = atomic weight, Dn = density'''


# In[3]:

'''Imports numpy, and matplotlib libaries 
numpy & matplotlib allow for complex math'''

import numpy as np
import matplotlib.pyplot as plt


# In[4]:

def MgZnCaAtToWt(MgAt = 65., ZnAt = 30.):
'''Creates an object array of a Mg, Zn, Ca alloy
At% of Mg and Zn are inputs
Defaults are Mg 65, Zn 30, which gives Ca 5'''

'assertion test for invalid inputs'
assert MgAt + ZnAt < 100.0, 'At% of Mg plus Zn must be less than 100%'
assert (MgAt and ZnAt) > 0.0, 'At% of Mg and Zn must be greater than 0'

return [
{'name': 'Mg', 
	'atomicPercent': round(MgAt, 3), #alloy atomic percent
	'atomicWeight': 24.305, #amu
	'density': 1.738}, #g/cm3

{'name': 'Zn', 
	'atomicPercent': round(ZnAt, 3), #alloy atomic percent
	'atomicWeight': 65.382, #amu
	'density': 7.14}, #g/cm3

{'name': 'Ca', 
	'atomicPercent': round((100. - MgAt - ZnAt), 3), #alloy atomic percent
	'atomicWeight': 40.078, #amu
	'density': 1.55} #g/cm3
]


# In[5]:

def AtToWtCalc(Mg = 65, Zn = 30, totalAlloyMass = 100.0):
'''Loop to calculate MgZnCa Wt% of alloy elements from At%
Mg = At% of Mg, default 65 at%
Zn = At% of Zn, default 30 at%
totalAlloyMass = total mass of the charge, default 100 grams'''

'assertion test for invalid inputs'
assert Mg + Zn < 100.0, 'At% of Mg plus Zn must be less than 100%'
assert (Mg and Zn) > 0.0, 'At% of Mg and Zn must be greater than 0'
assert totalAlloyMass > 0.0, 'The total alloy charge mass must be greater than 0.0 grams'

'Stores the analysis alloy within the Object Array function'
elements = MgZnCaAtToWt(Mg, Zn)

'''Returns alloy name and atomic percent.
AlloyArray creates a string array of Name+At% via a loop 
%s%.xf formats the string; first item str, second number of x floats'''
AlloyArray = ['%s%.1f' % (i['name'], i['atomicPercent']) for i in elements]
'join combines AlloyArray into a single string, seperatred by the Delimiter'
delimiter = ' '
AlloyAt = delimiter.join(AlloyArray)  
print 'For an alloy of ', AlloyAt, 'at%'

'''Loop to return At% to Wt% denominator of 
total sum of atomic percent (at%) times atomic weight (AMU) of alloy
i.e. C1'*A1 + ... + Cn'*An
Where Cn = wt%, Cn' = at%, An = atomic weight'''
totalAtxAMU = 0
for i in elements:
totalAtxAMU += i['atomicPercent'] * i['atomicWeight']

'''Loop to return alloy density
Uses start of main loop to computer Wt% for use in calculation
i.e. 1/D = C1/D1 + ... + Cn/Dn'''
OneOnDensity = 0
for i in elements:
AtxAMU = i['atomicPercent'] * i['atomicWeight']
wt = AtxAMU / totalAtxAMU
OneOnDensity += (wt / i['density'])
alloyDensity = 1. / OneOnDensity
print 'The expected alloy density is', round(alloyDensity, 3), 'g/cm^3'
print

'''Main loop to computer Wt%, mass of each element, and alloy density
i.e. C1 = C1' * A1 / (C1' * A1 + ... + Cn' * An)'''
OneOnDensity = 0
for i in elements:
AtxAMU = i['atomicPercent'] * i['atomicWeight']
wt = AtxAMU / totalAtxAMU
wt100 = 100 * wt
wtRound = round(wt100, 3)
print i['name'], i['atomicPercent'], 'at% is ', wtRound, 'wt%'

'Alloy weight calcs happen here'
AlloyMass = totalAlloyMass * wt
AlloyMassRound = round(AlloyMass, 3) 
print AlloyMassRound, 'grams', i['name'], 'per', totalAlloyMass, 'gram charge'
print 


# In[6]:

def MgZnCaMass(MgWt = 42.222, ZnWt = 52.422, CaWt = 5.356):
'''Calculate Wt% of given masses of Mg, Zn, and C
Creates an object array of those Mg, Zn, Ca Wt%
Mass of Mg, Zn, and Ca are inputs
Defaults are the masses for a Mg65Zn30Ca5 At% alloy'''

'assertion test for invalid inputs'
assert (MgWt and ZnWt and CaWt) > 0.0, 'Wt of elements must be greater than 0'

'Works out wt% of alloy from componet input masses'
TotalWeight = float(MgWt + ZnWt + CaWt)
MgWtPer = 100 * (MgWt / TotalWeight)
ZnWtPer = 100 * (ZnWt / TotalWeight)
CaWtPer = 100 * (CaWt / TotalWeight)

return [
{'name': 'Mg', 
	'weightPercent': round(MgWtPer, 3), #alloy atomic percent
	'atomicWeight': 24.305, #amu
	'density': 1.738}, #g/cm3

{'name': 'Zn', 
	'weightPercent': round(ZnWtPer, 3), #alloy atomic percent
	'atomicWeight': 65.382, #amu
	'density': 7.14}, #g/cm3

{'name': 'Ca', 
	'weightPercent': round(CaWtPer, 3), #alloy atomic percent
	'atomicWeight': 40.078, #amu
	'density': 1.55} #g/cm3
]
\end{python}

\newpage
\subsubsection{SMG Target Deposition Calculations}

\begin{python}
# In[1]:

'''Code for the Deposition of Targets
Code by Scott Gleason, University of New South Wales (UNSW), Australia 
S.Gleason@student.unsw.edu.au, May 2015'''


# In[2]:

'imports numpy and matplotlib libaries into code'
import numpy as np
import matplotlib.pyplot as plt


# In[3]:

def DepositionTime(Rate = 1.5, Thickness = 1000):
'''Finds the deposition time for a desidered film thickness 
time = Thickness / Rate
Rate is in nm/s, default 1.5 nm/s
Thickness is in nm, default 1000 nm'''

assert (Rate and Thickness) > 0.0, "Rate/Thickness must be greater than 0.0"

TimeSeconds = Thickness / Rate #seconds
TimeMinutes = TimeSeconds / 60. #minutes
print 'Deposition Rate is', Rate, 'nm/s'
print 'Film Thickness T is', Thickness, 'nm'
print 'Time to deposite thin film', round(TimeSeconds, 3), 'seconds'
print 'Time to deposite thin film', round(TimeMinutes, 3), 'minutes'


# In[4]:

def SubstrateTemp(GlassTransition = 132, LowerTsub = 0.7, UpperTsub = 0.8):
'''Finds the USG deposition temperatures 
Tsub = GlassTransition * LowerTsub OR UpperTsub
GlassTransition is in C, default 132 C
LowerTsub & UpperTsub are unitless, default is 0.7 and 0.8'''

assert (GlassTransition and LowerTsub and UpperTsub) > 0.0, "Parameters must be greater than 0.0"

TempKelvin = GlassTransition + 273.15 #kelvin
LowerTsubKelvin = LowerTsub * TempKelvin #kelvin
UpperTsubKelvin = UpperTsub * TempKelvin #kelvin
LowerTsubCelius = LowerTsubKelvin - 273.15 #celious
UpperTsubCelius = UpperTsubKelvin - 273.15 #celious

print 'The glass transition temperature (Tg) is', GlassTransition, 'C,  or', TempKelvin, 'K'
print 'The lower Tsub limit is', LowerTsub, 'of Tg'
print 'The upper Tsub limit is', UpperTsub, 'of Tg'
print 'The lower Tsub temperature is', LowerTsubCelius, 'C,  or', LowerTsubKelvin, 'K'
print 'The upper Tsub temperature is', UpperTsubCelius, 'C,  or', UpperTsubKelvin, 'K'

\end{python}

\newpage
\subsubsection{Casting Volume Calculations}

\begin{python}
# In[1]:

'''Code for the Volume of Tagets
Code by Scott Gleason, University of New South Wales (UNSW), Australia 
S.Gleason@student.unsw.edu.au, April 2015'''


# In[2]:

'imports numpy and matplotlib libaries into code'
import numpy as np
import matplotlib.pyplot as plt


# In[3]:

def CylindarVolume(Diameter = 25.4, Thickness = 4.0):
'''Finds the volume of a cylindar as given by 
V = pi * (D/2)^2 * t
Diameter is in mm, default 25.4 mm
Thickness is in mm, default 4 mm'''

assert Diameter > 0.0, "Diameter must be a positive value"
assert Thickness > 0.0, "Thickness must be a positive value"

Radius = Diameter / 2.0
Volume = np.pi * Thickness * Radius**2 #mm^3
VolumeCC = Volume / 1000 #cc #(1000mm^3 per cm^3)
print 'Radius R is', Radius, 'mm'
print 'Thickness t is', Thickness, 'mm'
print 'Volume of cylinder is', round(Volume, 3), 'mm^3'
print 'Volume of cylinder is', round(VolumeCC, 3), 'cc'


# In[4]:

def PlateVolume(Thickness = 4.0, Height = 49.0, Length = 102.0):
'''Finds the volume of a plate as given by 
V = L * H * t
Thickness is in mm, default 4 mm
Height is in mm, default 49 mm
Length is in mm, default 102 mm'''

assert Thickness > 0.0, "Thickness must be a positive value"
assert Height > 0.0, "Height must be a positive value"
assert Length > 0.0, "Length must be a positive value"

Volume = Length * Height * Thickness #mm^3
VolumeCC = Volume / 1000 #cc #(1000mm^3 per cm^3)
print 'Thickness t is', Thickness, 'mm'
print 'Height H is', Height, 'mm'
print 'Length L is', Length, 'mm'
print 'Volume of plate is', round(Volume, 3), 'mm^3'
print 'Volume of plate is', round(VolumeCC, 3), 'cc'
return round(VolumeCC, 3)


# In[5]:

def PlateRiserVolume(Thickness = 4.0, RiserHeight = 18.0, Length = 102.0):
'''Finds the volume of a plate riser as given by 
V = L * Rh/2 * (t0 + Thickness)
Thickness is in mm, default 4 mm
RiserHeight is in mm, default 18 mm
Length is in mm, default 102 mm'''

assert Thickness > 0.0, "Thickness must be a positive value"
assert RiserHeight > 0.0, "RiserHeight must be a positive value"
assert Length > 0.0, "Length must be a positive value"

OpeningThickness = 13.0 #Thickness of plate riser opening
Area = RiserHeight/2 * (Thickness + OpeningThickness) #mm^2
Volume = Length * Area #mm^3
VolumeCC = Volume / 1000 #cc #(1000mm^3 per cm^3)
print 'Riser Base Thickness t0 is', Thickness, 'mm'
print 'Riser Opening Thickness t is', Thickness + OpeningThickness, 'mm'
print 'Riser Height Rh is', RiserHeight, 'mm'
print 'Length L is', Length, 'mm'
print 'Volume of riser is', round(Volume, 3), 'mm^3'
print 'Volume of riser is', round(VolumeCC, 3), 'cc'
return round(VolumeCC, 3)


# In[6]:

def VolumeCCToMass(VolumeCC = 8.107, Density = 2.85):
'''Converts cc volume to total mass from density 
VolumeCC = volume in cc, default to 8.107 cc
Density = Total alloy density, default to 2.85 g/cm3 (Mg65Zn30Ca5)'''

assert VolumeCC > 0.0, "Volume must be a positive value"
assert Density > 0.0, "Density must be a positive value"

Mass = VolumeCC * Density #cc * g/cm3 = g
print VolumeCC, 'cc of material with a density of', Density, 'g/cm3 is:'
print round(Mass,3), 'g'


# In[7]:

def MassToVolumeCC(Mass = 100.0, Density = 2.85):
'''Converts total mass to cc volume from density 
Mass = Total alloy mass, default to 100 g
Density = Total alloy density, default to 2.85 g/cm3 (Mg65Zn30Ca5)'''

assert Mass > 0.0, "Mass must be a positive value"
assert Density > 0.0, "Density must be a positive value"

VolumeCC = Mass / Density #g / (g/cm3) = cm3 = cc
print Mass, 'g of material with a density of', Density, 'g/cm3 is:'
print round(VolumeCC,3), 'cc'


# In[8]:

print '2-Inch Target size:'
CylindarVolume(50.8, 4.5)


# In[9]:

print 'Small Crucible Size:'
CylindarVolume(32, 85)


# In[10]:

print 'Large Crucible Size:'
CylindarVolume(34, 85)

\end{python}

\newpage
\section{Plotting Codes}
\subsubsection{MgZnCa System Property Plots}

\begin{python}
# In[1]:

'''Code to plot MgZnCa Temperature Properites from .txt directory
Code by Scott Gleason, University of New South Wales (UNSW), Australia 
S.Gleason@student.unsw.edu.au, June 2015'''


# In[2]:

'''Imports numpy, matplotlib, and glob libaries 
numpy & matplotlib allow for complex math
glob allows for pattern matching with * and ? wild cards
(note glob's only function is glob (i.e. glob.glob('search critrea')))'''

'%matplotlib inline'
'Generates plots inline of the notebook. Commit out if want indivudual files'

import numpy as np
import matplotlib.pyplot as plt
from matplotlib.ticker import AutoMinorLocator #Used for minor axis ticks
import glob


# In[3]:

cd 'C:\\Users\\z3492622\\Documents\\PhD Project\\Results'


# In[4]:

cd 'Modelling\\'


# In[5]:

'Directs to directory containing files to be anaylised'

Folder = 'MgZnCa BMG Properites'

print 'Directory to be anaylised:'
print Folder


# In[6]:

def dataFilesRange(FileType, StartFile = 0, EndFile = None):
'''Preforms a glob list of *.filetype within directory
FileType = csv, txt, etc
StartFile = Number of first file to be anaylsis from the glob
Defaulted to 0
EndFile = Number of last file to be anaylsis from the glob
Defaulted to None, as to return last value'''

'assertion test for the function'
assert StartFile > -1, 'First file must be atleast 0'
if EndFile<>None:
assert EndFile > 0, 'Last file must be atleast 1'

dataFiles = glob.glob(Folder + '\\*.'+ FileType)

'set range of files to be analysed'
Range = dataFiles[StartFile:EndFile]

return Range


# In[7]:

def dataArray(filename, Delimiter = ',', StartRow = 1, LastColumn = None):
'''Creates an array from an input file, and defines limits of the array
filename = file to anaylisied
Delimiter = data separator (defaulted to ',' for .csv)
StartRow = row data starts on (defaulted to 1)
LastColumn = column data ends on (defaulted to None)
Note: .txt use 'None' as Delimiter'''

assert StartRow > 0, 'First row must be atleast 1'

'''If condition determining the last column of the array
also contains assertion to ensure last column is valid'''
if LastColumn<>None:
assert LastColumn > 0, 'Last column must be atleast 1'
Column = range(0, LastColumn, 1)
else:
Column = None

'Returns array of data, do not edit this string'
return np.loadtxt(filename, delimiter = Delimiter,
skiprows = StartRow - 1, 
usecols = Column)


# In[8]:

def plotFile(filename, title, xAxis=1, yAxis=2):
'''Generates a plot of data from a file, and formates chart layout
filename = file to anaylisied
title = tile of the plot
xAxis = Column of X-Axis Data (defaulted to 1st column)
yAxis = Column of Y-Axis Data (defaulted to 2nd column)'''

'Assertion to ensure x and y axis are valid'
assert xAxis > 0, 'X Axis must be at least 1'
assert yAxis > 0, 'Y Axis must be at least 1'

'Defines the columns X & Y axis data come from in filename array'
xAxisData = filename[:, xAxis-1]
Y1 = filename[:, yAxis-1]
Y2 = filename[:, yAxis]
Y3 = filename[:, yAxis+1]
Y4 = filename[:, yAxis+2]
Y5 = filename[:, yAxis+3]

'Define Plot layout, axis lables, legends, etc'
PlotLayout = {'fileName': title, # Plot filename 
	'title': title, # Title lable
	'titleFontSize': 25, # Title lable font size
	'X-Lable': '$X$ $(at\%)$', # X axis lable
	'Y-Lable': '$T$  $(^{\circ}C)$', # Y axis lable
	'axisFontSize': 18, # X & Y axis lable font size
	'xAxisMin': -1, # lower scale limit for x axis
	'xAxisMax': 20, # upper scale limit for X axis
	'yAxisMin': 0, # lower scale limit for y axis
	'yAxisMax': 500, # upper scale limit for y axis
	'lineColourType': 'b', # line colour and style
	'lineWeight': 1.2 # weight of line
}

'Add PlotLayout objects to plot'
plt.figure(PlotLayout['fileName'])
plt.title(PlotLayout['title'], fontsize = PlotLayout['titleFontSize'])
plt.xlabel(PlotLayout['X-Lable'], fontsize = PlotLayout['axisFontSize'])
plt.ylabel(PlotLayout['Y-Lable'], fontsize = PlotLayout['axisFontSize'])

'Turns on minor axis ticks'
plt.minorticks_on()

'Generates the plot'
plt.hold(True)
plt.plot(xAxisData, Y1, linewidth = PlotLayout['lineWeight'],
marker='^', label='$T_{g}$')
plt.plot(xAxisData, Y2, linewidth = PlotLayout['lineWeight'],
marker='v', label='$T_{x}$')
plt.plot(xAxisData, Y3, linewidth = PlotLayout['lineWeight'],
marker='s', label='$T_{m}$')
plt.plot(xAxisData, Y4, linewidth = PlotLayout['lineWeight'],
marker='D', label='$T_{l}$')
plt.plot(xAxisData, Y5, linewidth = PlotLayout['lineWeight'],
marker='p', label='$\Delta T$', ls = '--')
plt.hold(False)
plt.axis([PlotLayout['xAxisMin'], PlotLayout['xAxisMax'], 
PlotLayout['yAxisMin'], PlotLayout['yAxisMax']])
plt.legend()
plt.show()


# In[9]:

filesToPlot = dataFilesRange('txt', 1, 4)
print filesToPlot


# In[10]:

def PlotFile(filename, title):
'''Plot a .txt or .csv file with a title
Prints filepath of each *.filetype file anaylisied 
Uses dataArray() function to print matrix of each file
Uses dataArray() and plotFile() functions to plot each file'''

print filename #Prints files
'''dataArray takes 4 arrguments; filename, Delimiter, StartRow, and LastColumn
Note1: For .csv Delimiter = ',' and for .txt Delimiter = None'''
dataToPlot = dataArray(filename, None, 2, None) #Defines data and limits
print dataToPlot #Prints matrix
'plotfile requires the X & Y axis be specified'
plotFile(dataToPlot, title, 1, 2) #Generates plot


# In[11]:

PlotFile(filesToPlot[0], '$Mg_{70-X}Zn_{24+X}Ca_{6}$')


# In[12]:

PlotFile(filesToPlot[1], '$Mg_{72-X}Zn_{24+X}Ca_{4}$')


# In[13]:

PlotFile(filesToPlot[2], '$Mg_{75-X}Zn_{20+X}Ca_{5}$')
\end{python}

\newpage
\subsubsection{Gibbs Free Energy Plots}

\begin{python}
# In[1]:

'''Code to plot Gibbs Free Energy Schematic Plot
Code by Scott Gleason, University of New South Wales (UNSW), Australia 
S.Gleason@student.unsw.edu.au, May 2015'''


# In[2]:

'''Imports numpy, matplotlib, and glob libaries 
numpy & matplotlib allow for complex math'''

'%matplotlib inline'
'Generates plots inline of the notebook. Commit out if want indivudual files'

import numpy as np
import matplotlib.pyplot as plt
from matplotlib.ticker import AutoMinorLocator #Used for minor axis ticks


# In[3]:

def SystemFreeEnergy():
'''Plot of Free Energy for differnt system phases (Liquid, solid, glass)'''

'Define Plot layout, axis lables, legends, etc'
PlotLayout = {'fileName': 'Gibbs Free Energy', # Plot filename 
	'title': '', # Title lable
	'titleFontSize': 20, # Title lable font size
	'X-Lable': '$Temperature$', # X axis lable
	'Y-Lable': '$G$', # Y axis lable
	'axisFontSize': 25, # X & Y axis lable font size
	'xAxisMax': 55, # upper scale limit for X axis
	'yAxisMax': 85, # upper scale limit for y axis
	'lineWeight': 2.5 # weight of line
}

'Add PlotLayout objects to plot'
plt.figure(PlotLayout['fileName'])
plt.title(PlotLayout['title'], fontsize = PlotLayout['titleFontSize'])
plt.xlabel(PlotLayout['X-Lable'], fontsize = PlotLayout['axisFontSize'])
plt.ylabel(PlotLayout['Y-Lable'], fontsize = PlotLayout['axisFontSize'])

'''Add Free Energy Lines
Note: Weakly bonded structures have greater slopes (large Cp/T dT value)
Assume typical glass bonding Ionic 
Assume typical solid bonding Covalent
Covalent > Ionic bond strength. Thus solid slope < glass''' 
FreeEnergy = [
{'x0': 0, 'x1': 100, 'y0': 80, 'y1': 0, 
	'colour': 'r', 'line': '-', 'label': 'Liquid'},
{'x0': 0, 'x1': 100, 'y0': 58, 'y1': 33,
	'colour': 'g', 'line': '-', 'label': 'Solid'},
{'x0': 0, 'x1': 100, 'y0': 70, 'y1': 40,
	'colour': 'b', 'line': '--', 'label': 'Glass'}]

for i in FreeEnergy:
plt.plot([i['x0'], i['x1']], [i['y0'], i['y1']],
color=i['colour'], linestyle=i['line'], 
linewidth=PlotLayout['lineWeight'],
label=i['label'])

'Add annotations to the plot'
PlotAnnotate = [
{'label': '$T_{m}$', 'xPos': 40, 'yPos1': 37, 'yPos2': 63, 'colour': 'k'},
{'label': '$T_{g}$', 'xPos': 20, 'yPos1': 39, 'yPos2': 70, 'colour': 'k'}]

for i in PlotAnnotate:
plt.annotate(i['label'], size = 20, xy=(i['xPos'], i['yPos2']),
xytext=(i['xPos'], i['yPos1']),
arrowprops=dict(arrowstyle="-", linewidth=1,
color=i['colour']),
horizontalalignment='center')

'Generates the plot'
plt.axis([0, PlotLayout['xAxisMax'], 30, PlotLayout['yAxisMax']])
ax = plt.subplot(1, 1, 1)
ax.set_xticklabels([]) #Removes xAxis numbers
ax.set_yticklabels([]) #Removes yAxis numbers
plt.legend() #Add legend
plt.show()


# In[4]:

def NucleationEnergy():
'''Plot of Gibbs Nucleation Energy
G(r)=(4 pi r^2)SL + (4/3) pi r^3(GV)
Where: Surface Energy SL = 1
Where: Volume Energy GV = -1'''

'Define Plot layout, axis lables, legends, etc'
PlotLayout = {'fileName': 'Nucleation Free Energy', # Plot filename 
	'title': '', # Title lable
	'titleFontSize': 20, # Title lable font size
	'X-Lable': '$r$', # X axis lable
	'Y-Lable': '$G$', # Y axis lable
	'axisFontSize': 30, # X & Y axis lable font size
	'xAxisMax': 3.7, # upper scale limit for X axis
	'yAxisMax': 25, # upper scale limit for y axis
	'yAxisMin': -15, # lower scale limit for y axis
	'lineWeight': 3 # weight of line
}

'Add PlotLayout objects to plot'
plt.figure(PlotLayout['fileName'])
plt.title(PlotLayout['title'], fontsize = PlotLayout['titleFontSize'])
plt.xlabel(PlotLayout['X-Lable'], fontsize = PlotLayout['axisFontSize'])
plt.ylabel(PlotLayout['Y-Lable'], fontsize = PlotLayout['axisFontSize'])

'Add Nucleation Function and lines'
r = np.linspace(0,5,1000) # linearly spaced numbers
SL = 4. * np.pi * r**2 #Surface Energy
GV = -4./3 * np.pi * r**3 #Volume Energy
G = SL + GV #Total Nucleation Energy

NucleationLines = [
{'xAxis': r, 'yAxis': SL, 'colour': 'r', 'line': '--',
	'label': '$(4 \pi r^{2}) \gamma_{SL}$'},
{'xAxis': r, 'yAxis': GV, 'colour': 'g', 'line': '--',
	'label': '$(4/3) \pi r^{3} \Delta G_{V}$'},
{'xAxis': r, 'yAxis': G, 'colour': 'b', 'line': '-',
	'label': '$\Delta G(r)$'}]

for i in NucleationLines:
plt.plot(i['xAxis'], i['yAxis'], 
color=i['colour'], linestyle=i['line'], 
linewidth=PlotLayout['lineWeight'],
label=i['label'])

'Add annotations to the plot'
PlotAnnotate = [
{'label': '$\Delta G{*}$', 'xPos': 2, 'yPos1': 17.3, 'yPos2': 0,
	'colour': 'k', 'size': 18},
{'label': '$r{*}$', 'xPos': 2, 'yPos1': -3, 'yPos2': 0,
	'colour': 'k', 'size': 25}]

for i in PlotAnnotate:
plt.annotate(i['label'], size = i['size'], xy=(i['xPos'], i['yPos2']),
xytext=(i['xPos'], i['yPos1']),
arrowprops=dict(arrowstyle="-", linewidth=0.75,
color=i['colour']),
horizontalalignment='center')

'Generates the plot'
plt.axis([0, PlotLayout['xAxisMax'],
PlotLayout['yAxisMin'], PlotLayout['yAxisMax']])
plt.annotate('',xy=(3.5, 0), xytext=(0,0),
arrowprops=dict(arrowstyle="->",linewidth=1.5,color='k'))
ax = plt.subplot(1, 1, 1)
ax.set_xticklabels([]) #Removes xAxis numbers
ax.set_yticklabels([]) #Removes yAxis numbers
plt.legend() #Add legend
plt.show()


# In[5]:

SystemFreeEnergy()


# In[6]:

NucleationEnergy()

\end{python}


\newpage
\subsubsection{EDS Plotting Loop Function}

\begin{python}
# In[1]:

'''Code to plot EDS spectra anaylsis from a master .csv or .txt directory
Code by Scott Gleason, University of New South Wales (UNSW), Australia 
S.Gleason@student.unsw.edu.au, April 2015'''

'''Future Devlopment - Make a derivative and intergration of the dataMatrix,
and plot all 3 Matrixs in a row for each file (see SoftwareCap). 
This will be for plotting DSC / SMG data'''


# In[2]:

'''Imports numpy, matplotlib, and glob libaries 
numpy & matplotlib allow for complex math
glob allows for pattern matching with * and ? wild cards
(note glob's only function is glob (i.e. glob.glob('search critrea')))'''

get_ipython().magic(u'matplotlib inline')
'Generates plots inline of the notebook. Commit out if want indivudual files'

import numpy as np
import matplotlib.pyplot as plt
from matplotlib.ticker import AutoMinorLocator #Used for minor axis ticks
import glob


# In[3]:

cd 'C:\\Users\\z3492622\\Documents\\PhD Project\\Results'


# In[4]:

cd 'SEM\\2015-04-14 (T1 2 3 5)\\'


# In[5]:

'Directs to directory containing files to be anaylised'

Folder = 'EDS Spectra Sheets'

print 'Directory to be anaylised:'
print Folder


# In[6]:

def dataFilesRange(FileType, StartFile = 0, EndFile = None):
'''Preforms a glob list of *.filetype within directory
FileType = csv, txt, etc
StartFile = Number of first file to be anaylsis from the glob
Defaulted to 0
EndFile = Number of last file to be anaylsis from the glob
Defaulted to None, as to return last value'''

'assertion test for the function'
assert StartFile > -1, 'First file must be atleast 0'
if EndFile<>None:
assert EndFile > 0, 'Last file must be atleast 1'

dataFiles = glob.glob(Folder + '\\*.'+ FileType)

'set range of files to be analysed'
Range = dataFiles[StartFile:EndFile]

return Range


# In[7]:

def dataArray(filename, Delimiter = ',', StartRow = 1, LastColumn = None):
'''Creates an array from an input file, and defines limits of the array
filename = file to anaylisied
Delimiter = data separator (defaulted to ',' for .csv)
StartRow = row data starts on (defaulted to 1)
LastColumn = column data ends on (defaulted to None)
Note: .txt use 'None' as Delimiter'''

assert StartRow > 0, 'First row must be atleast 1'

'''If condition determining the last column of the array
also contains assertion to ensure last column is valid'''
if LastColumn<>None:
assert LastColumn > 0, 'Last column must be atleast 1'
Column = range(0, LastColumn, 1)
else:
Column = None

'Returns array of data, do not edit this string'
return np.loadtxt(filename, delimiter = Delimiter,
skiprows = StartRow - 1, 
usecols = Column)


# In[8]:

def plotFile(filename, xAxis=1, yAxis=2):
'''Generates a plot of data from a file, and formates chart layout
filename = file to anaylisied
xAxis = Column of X-Axis Data (defaulted to 1st column)
yAxis = Column of Y-Axis Data (defaulted to 2nd column)'''

'Assertion to ensure x and y axis are valid'
assert xAxis > 0, 'X Axis must be at least 1'
assert yAxis > 0, 'Y Axis must be at least 1'

'Defines the columns X & Y axis data come from in filename array'
xAxisData = filename[:, xAxis-1]
yAxisData = filename[:, yAxis-1]

'Define Plot layout, axis lables, legends, etc'
PlotLayout = {'fileName': 'Spectra Counts', # Plot filename 
	'title': 'Counts / $keV$', # Title lable
	'titleFontSize': 20, # Title lable font size
	'X-Lable': '$keV$', # X axis lable
	'Y-Lable': 'Counts', # Y axis lable
	'axisFontSize': 16, # X & Y axis lable font size
	'xAxisMax': 20, # upper scale limit for X axis
	'yAxisMax': 8000, # upper scale limit for y axis
	'lineColourType': 'k', # line colour and style
	'lineWeight': 0.8 # weight of line
}

'Add PlotLayout objects to plot'
plt.figure(PlotLayout['fileName'])
plt.title(PlotLayout['title'], fontsize = PlotLayout['titleFontSize'])
plt.xlabel(PlotLayout['X-Lable'], fontsize = PlotLayout['axisFontSize'])
plt.ylabel(PlotLayout['Y-Lable'], fontsize = PlotLayout['axisFontSize'])

'Add annotations to the plot'
PlotAnnotate = [
{'label': 'Mg', 'xPos': 1.253, 'yPos': 4000, 'colour': 'g'},
{'label': 'Ca', 'xPos': 0.341, 'yPos': 3000, 'colour': 'c'},
{'label': 'Ca', 'xPos': 3.690, 'yPos': 3000, 'colour': 'c'},
{'label': 'Zn', 'xPos': 1.012, 'yPos': 7500, 'colour': 'm'},
{'label': 'Zn', 'xPos': 8.630, 'yPos': 2000, 'colour': 'm'},
{'label': '', 'xPos': 1.302, 'yPos': 500, 'colour': 'g'},
{'label': '', 'xPos': 4.012, 'yPos': 500, 'colour': 'c'},
{'label': '', 'xPos': 0.345, 'yPos': 500, 'colour': 'c'},
{'label': '', 'xPos': 9.570, 'yPos': 500, 'colour': 'm'},
{'label': '', 'xPos': 1.043, 'yPos': 500, 'colour': 'm'}]

for i in PlotAnnotate:
plt.annotate(i['label'], xy=(i['xPos'], 0),
xytext=(i['xPos'], i['yPos']),
arrowprops=dict(arrowstyle="-", linewidth = 0.4,
color=i['colour']),
horizontalalignment='center')

'Turns on minor axis ticks'
#plt.minorticks_on()
ax = plt.subplot(1, 1, 1)
ax.xaxis.set_minor_locator(AutoMinorLocator(5))

'Generates the plot'
plt.plot(xAxisData, yAxisData,
PlotLayout['lineColourType'], 
linewidth = PlotLayout['lineWeight'])
plt.axis([0, PlotLayout['xAxisMax'], 0, PlotLayout['yAxisMax']])
plt.show()


# In[9]:

filesToPlot = dataFilesRange('txt', 0, 5)
print filesToPlot


# In[10]:

def PlotFilesLoop():
'''For loop to print dataFileRange
Prints filepath of each *.filetype file anaylisied 
Uses dataArray() function to print matrix of each file
Uses dataArray() and plotFile() functions to plot each file'''

for files in filesToPlot:
print files #Prints files

'''dataArray takes 4 arrguments; filename, Delimiter, StartRow, and LastColumn
Note1: For .csv Delimiter = ',' and for .txt Delimiter = None'''
dataToPlot = dataArray(files, None, 26, 2) #Defines data and limits
print dataToPlot #Prints matrix
'plotfile requires the X & Y axis be specified'
plotFile(dataToPlot, 1, 2) #Generates plot


# In[11]:

PlotFilesLoop()

\end{python}

\end{document}